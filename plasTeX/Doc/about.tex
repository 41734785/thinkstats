
This document was writted using LaTeX (\url{http://www.latex-project.org/}).  
The documents use macros written for documenting the Python 
(\url{http://www.python.org}) language and Python packages.  Generating
the PDF version of the document is simply a matter of using the 
\program{pdflatex} command.  Generating the HTML version of the document,
of course, uses plasTeX.  

The wonderful thing about the HTML version is that it was generated from
the LaTeX source and Python style files without customization\footnote{
Ok, there was one customization to \macro{var} for a whitespace issue,
but the change works both in the PDF and HTML version}!  In fact, in its
current state, plasTeX can generate the HTML versions of the Python
documentation found on their website, \url{http://www.python.org/doc/}.
Without customization of plasTeX, the only remaining issues are that the
module index is missing and there are some formatting differences.
Not bad, considering plasTeX is doing actually expanding the LaTeX
document natively.
