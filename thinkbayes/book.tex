% LaTeX source for ``Think Bayes: Bayesian Statistics Made Simple''
% Copyright 2012  Allen B. Downey.

% License: Creative Commons Attribution-NonCommercial 3.0 Unported License.
% http://creativecommons.org/licenses/by-nc/3.0/
%

\documentclass[12pt]{book}
\usepackage[width=5.5in,height=8.5in,
  hmarginratio=3:2,vmarginratio=1:1]{geometry}

% for some of these packages, you might have to install
% texlive-latex-extra (in Ubuntu)

\usepackage[T1]{fontenc}
\usepackage{textcomp}
\usepackage{mathpazo}
\usepackage{url}
\usepackage{fancyhdr}
\usepackage{graphicx}
\usepackage{subfig}
\usepackage{amsmath}
\usepackage{amsthm}
\usepackage{makeidx}
\usepackage{setspace}
\usepackage{hevea}                           
\usepackage{upquote}

\title{Think Bayes}
\author{Allen B. Downey}

\newcommand{\thetitle}{Think Bayes: Bayesian Statistics Made Simple}
\newcommand{\theversion}{0.1.0}

% these styles get translated in CSS for the HTML version
\newstyle{a:link}{color:black;}
\newstyle{p+p}{margin-top:1em;margin-bottom:1em}
\newstyle{img}{border:0px}

% change the arrows in the HTML version
\setlinkstext
  {\imgsrc[ALT="Previous"]{back.png}}
  {\imgsrc[ALT="Up"]{up.png}}
  {\imgsrc[ALT="Next"]{next.png}} 

\makeindex

\newif\ifplastex
\plastexfalse

\begin{document}

\frontmatter

\ifplastex

\else
\fi

\newcommand{\PMF}{\mathrm{PMF}}
\newcommand{\PDF}{\mathrm{PDF}}
\newcommand{\CDF}{\mathrm{CDF}}
\newcommand{\ICDF}{\mathrm{ICDF}}

\ifplastex
    \usepackage{localdef}
    \maketitle

\else

\input{latexonly}

\begin{latexonly}

\renewcommand{\blankpage}{\thispagestyle{empty} \quad \newpage}

% TITLE PAGES FOR LATEX VERSION

%-half title--------------------------------------------------
\thispagestyle{empty}

\begin{flushright}
\vspace*{2.0in}

\begin{spacing}{3}
{\huge Think Bayes: Bayesian Statistics Made Simple}\\
{\Large }
\end{spacing}

\vspace{0.25in}

Version \theversion

\vfill

\end{flushright}

%--verso------------------------------------------------------

\blankpage
\blankpage

%--title page--------------------------------------------------
\pagebreak
\thispagestyle{empty}

\begin{flushright}
\vspace*{2.0in}

\begin{spacing}{3}
{\huge Think Bayes}\\
{\Large Bayesian Statistics Made Simple}
\end{spacing}

\vspace{0.25in}

Version \theversion

\vspace{1in}


{\Large
Allen B. Downey\\
}


\vspace{0.5in}

{\Large Green Tea Press}

{\small Needham, Massachusetts}

\vfill

\end{flushright}


%--copyright--------------------------------------------------
\pagebreak
\thispagestyle{empty}

Copyright \copyright ~2012 Allen B. Downey.


\vspace{0.2in}

\begin{flushleft}
Green Tea Press       \\
9 Washburn Ave \\
Needham MA 02492
\end{flushleft}

Permission is granted to copy, distribute, and/or modify this document
under the terms of the Creative Commons Attribution-NonCommercial 3.0 Unported
License, which is available at \url{http://creativecommons.org/licenses/by-nc/3.0/}.

\vspace{0.2in}

\end{latexonly}


% HTMLONLY

\begin{htmlonly}

% TITLE PAGE FOR HTML VERSION

{\Large \thetitle}

{\large Allen B. Downey}

Version \theversion

\vspace{0.25in}

Copyright 2012 Allen B. Downey

\vspace{0.25in}

Permission is granted to copy, distribute, and/or modify this document
under the terms of the Creative Commons Attribution-NonCommercial 3.0
Unported License, which is available at
\url{http://creativecommons.org/licenses/by-nc/3.0/}.

\setcounter{chapter}{-1}

\end{htmlonly}

\fi
% END OF THE PART WE SKIP FOR PLASTEX

\chapter{Preface}
\label{preface}

\section*{}


Allen B. Downey \\*
Needham MA \\*

Allen B. Downey is a Professor of Computer Science at 
the Franklin W. Olin College of Engineering.




%\section*{Acknowledgements}



\section*{Contributor List}

If you have a suggestion or correction, please send email to 
{\tt downey@allendowney.com}.  If I make a change based on your
feedback, I will add you to the contributor list
(unless you ask to be omitted).
\index{contributors}

If you include at least part of the sentence the
error appears in, that makes it easy for me to search.  Page and
section numbers are fine, too, but not quite as easy to work with.
Thanks!

\small

\begin{itemize}

\item No contributors yet.

% ENDCONTRIB

\end{itemize}

\normalsize

\clearemptydoublepage

% TABLE OF CONTENTS
\begin{latexonly}

\tableofcontents

\clearemptydoublepage

\end{latexonly}

% START THE BOOK
\mainmatter

\newcommand{\p}{{\mathrm p}}
\newcommand{\and}{{\mathrm and}}
\newcommand{\not}{{\mathrm not}}


\chapter{Bayes's Theorem}
\label{intro}

\section{Conditional probability}

The fundamental idea behind all Bayesian statistics is Bayes's Theorem.
which is surprisingly easy to derive, provided that you understand
conditional probability.  So let's start with probability, then
conditional probabilty, and then we'll get back to 

A probability is a number between 0 and 1 (including both) that
represents a degree of belief in a fact or prediction.  A probability
of 1 represents certainty that a fact is true, or that a prediction
will come to pass.  A probability of 0 represents equal certainty
that the fact is false.

Intermediate values represent degrees of certainty.  The value 0.5,
often represented as 50\%, means that a predicted outcome is
as likely to happen as not.  For example, the probability that a tossed
coin lands face up is very close to 50\%.

A conditional probability is a probability based on some background
information.  For example, I might be interested in the probability
that I will have a heart attack in the next year.  According to the
CDC, ``Every year about 785,000 Americans have a first coronary attack.
(\url{http://www.cdc.gov/heartdisease/facts.htm})''

The U.S. population is about 311 million,
is roughly million, so the probability that a randomly-chosen
American will have a heart attack in the next year is 0.3\%.

But I am not a randomly-chosen American.  
Epidemiologists have identified many factors
affect the risk of heart attacks; depending on those factors,
my risk might be higher or lower than average.

I am male, 45 years old, and I have
borderline high cholesterol.  Those factors increase my chances.
However, I have low blood pressure and I don't smoke, and
those factors decrease my chances.

Plugging everything into the online calculator at
\url{http://hp2010.nhlbihin.net/atpiii/calculator.asp}, I find that my
risk of a heart attack in the next year is about 0.2\%, slightly
less than the national average.
That value is a conditional probability, because it is based on
a number of factors that make up my ``condition.''

The usual notation for conditional probability is $\p(A|B)$, which
is the probability of $A$ given that $B$ is true.  In this
example, $A$ represents the prediction that I will have a heart
attack in the next year, and $B$ is the set of conditions I listed.

Simple example with numbers?

\section{Conjunction}

``Conjoint probability'' is a fancy way to say the probability that
two things are both true.  I will write $\p(A \and B)$ to mean the
probability that $A$ and $B$ are both true.

If you learned about probability in the context of coin tosses and
dice, you might have learned the formula

\[ \p(A \and B) = \p(A) \p(B) \]

For example, if I toss two coins, and $A$ means the first coin lands
face up, and $B$ means the second coin lands face up, then $\p(A) =
\p(B) = 0.5$, and sure enough, $\p(A \and B) = \p(A) \p(B) = 0.25$.

But this formula only works because in this case $A$ and $B$ are
independent; that is, the second event does not depend on the first.
If you tell me the first coin is heads, that does not change $\p(B)$.

Here is a different example where the events are not independent.
Suppose again that $A$ means the first coin lands face up, but let's
add $C$, which means that {\em both} coins land face up.  What is the
probability of both events, $\p(A \and C)$?

These events are dependent because if we know whether or not $A$ is
true, that changes $\p(C)$.  In this case, if $A$ is true then $\p(C)
= 0.5$; but if $A$ is false $\p(C) = 0$.

To be more precise, I should not say that $\p(C)$ changes, but
rather that the conditional probabilities are different: $\p(C|A) = 0.5$
and $\p(C| \not A) = 0$.


Conjunction fallacy?





\end{document}

