% LaTeX source for ``Think Stats:
% Probability and Statistics for Programmers''
% Copyright 2010  Allen B. Downey.

% License: Creative Commons Attribution-Share Alike 3.0 Unported
% http://creativecommons.org/licenses/by-sa/3.0/
%

%\documentclass[10pt,b5paper]{book}
\documentclass[10pt]{book}
\usepackage[width=5.5in,height=8.5in,
  hmarginratio=3:2,vmarginratio=1:1]{geometry}

% for some of these packages, you might have to install
% texlive-latex-extra (in Ubuntu)

\usepackage{pslatex}
\usepackage{url}
\usepackage{fancyhdr}
\usepackage{graphicx}
\usepackage{amsmath, amsthm, amssymb}
\usepackage{exercise}                        % texlive-latex-extra
\usepackage{makeidx}
\usepackage{setspace}
\usepackage{hevea}                           
\usepackage{upquote}

\title{Think Stats}
\newcommand{\thetitle}{Think Stats: Probability and Statistics for Programmers}
\newcommand{\theversion}{1.0.0}

% these styles get translated in CSS for the HTML version
\newstyle{a:link}{color:black;}
\newstyle{p+p}{margin-top:1em;margin-bottom:1em}
\newstyle{img}{border:0px}

% change the arrows
\setlinkstext
  {\imgsrc[ALT="Previous"]{back.png}}
  {\imgsrc[ALT="Up"]{up.png}}
  {\imgsrc[ALT="Next"]{next.png}} 

\makeindex

\begin{document}

\frontmatter

% LATEXONLY

\input{latexonly}

\newtheorem{ex}{Exercise}[chapter]

\begin{latexonly}

\renewcommand{\blankpage}{\thispagestyle{empty} \quad \newpage}

%\blankpage
%\blankpage

% TITLE PAGES FOR LATEX VERSION

%-half title--------------------------------------------------
\thispagestyle{empty}

\begin{flushright}
\vspace*{2.0in}

\begin{spacing}{3}
{\huge Think Stats: Probability and Statistics for Programmers}\\
{\Large }
\end{spacing}

\vspace{0.25in}

Version \theversion

\vfill

\end{flushright}

%--verso------------------------------------------------------

\blankpage
\blankpage
%\clearemptydoublepage
%\pagebreak
%\thispagestyle{empty}
%\vspace*{6in}

%--title page--------------------------------------------------
\pagebreak
\thispagestyle{empty}

\begin{flushright}
\vspace*{2.0in}

\begin{spacing}{3}
{\huge Think Stats}\\
{\Large Probability and Statistics for Programmers}
\end{spacing}

\vspace{0.25in}

Version \theversion

\vspace{1in}


{\Large
Allen Downey\\
}


\vspace{0.5in}

{\Large Green Tea Press}

{\small Needham, Massachusetts}

%\includegraphics[width=1in]{figs/logo1.eps}
\vfill

\end{flushright}


%--copyright--------------------------------------------------
\pagebreak
\thispagestyle{empty}

{\small
Copyright \copyright ~2010 Allen Downey.


\vspace{0.2in}

\begin{flushleft}
Green Tea Press       \\
9 Washburn Ave \\
Needham MA 02492
\end{flushleft}

Permission is granted to copy, distribute, and/or modify this document
under the terms of the Creative Commons Attribution-Share Alike 3.0 Unported
License, which is available at \url{creativecommons.org/licenses/by-sa/3.0/}.

The original form of this book is \LaTeX\ source code.  Compiling this
code has the effect of generating a device-independent representation
of a textbook, which can be converted to other formats and printed.

The \LaTeX\ source for this book is available from
\url{www.thinkstats.com}.

The cover for this book is based on a photo by Paul Friel
(\url{flickr.com/people/frielp/}), who made it available under
the Creative Commons Attribution license.  The original photo
is at \url{flickr.com/photos/frielp/11999738/}.

\vspace{0.2in}

} % end small

\end{latexonly}


% HTMLONLY

\begin{htmlonly}

% TITLE PAGE FOR HTML VERSION

{\Large \thetitle}

{\large Allen B. Downey}

Version \theversion

\setcounter{chapter}{-1}

\end{htmlonly}


%\chapter{Preface}

%\section*{}



%Allen B. Downey \\
%Needham MA\\

%Allen Downey is an Associate Professor of Computer Science at 
%the Franklin W. Olin College of Engineering.




%\section*{Acknowledgements}



\section*{Contributor List}

\index{contributors}

If you have a suggestion or correction, please send email to 
{\tt downey@allendowney.com}.  If I make a change based on your
feedback, I will add you to the contributor list
(unless you ask to be omitted).

If you include at least part of the sentence the
error appears in, that makes it easy for me to search.  Page and
section numbers are fine, too, but not quite as easy to work with.
Thanks!

\small

\begin{itemize}

\item 

% ENDCONTRIB

\end{itemize}

\normalsize

\clearemptydoublepage

% TABLE OF CONTENTS
\begin{latexonly}

\tableofcontents

\clearemptydoublepage

\end{latexonly}

% START THE BOOK
\mainmatter


\chapter{Statistical thinking for programmers}

This book is about turning data into knowledge.  Data is cheap (at
least relatively); knowledge is harder to come by.

I will present three related pieces:

\begin{description}

\item[Probability] is the study of random events.  Most people have an
  intuitive understanding of degrees of probability, which is why we
  can use words like ``probably'' and ``unlikely'' without special
  training, but we will talk about how to make quantitative claims
  about those degrees.

\item[Statistics] is the discipline of using data samples to support
  claims about populations.  Most statistical analysis is based on
  probability, which is why these pieces are usually presented
  together.

\item[Computation] is a tool that is well-suited to quantitative
  analysis, and computers are commonly used to process statistics.
  Also (and more importantly for this book) computational experiments
  are useful for exploring concepts in probability and statistics.

\end{description}

The thesis of this book is that if you know how to program, you can
use that skill to help you understand probability and statistics.
These topics are often presented from a mathematical perspective, and
that approach works well for some people.  But some important ideas
in this area are hard to work with mathematically and relatively
easy to approach computationally.

Both approaches have merits, and the ideal might combine both, but
the goal of this book is to explore the computational path.

The rest of this chapter presents a case study motivated by a question
I heard when my wife and I were expecting our first child: do first
babies tend to arrive late?

\section{Do first babies arrive late?}

If you Google this question, you will find plenty of discussion.
Some people claim it's true, others say it's a myth, and some people
say it's the other way around: first babies come early.

In many of these discussions, people provide data to support their
claims.  I found many examples like these:

\begin{quote}

``My two friends that have given birth recently to their first babies,
BOTH went almost 2 weeks overdue before going into labour or being
induced.''

``My first one came 2 weeks late and now I think the second one is
going to come out two weeks early!!''

``I don't think that can be true because my sister was my mother's
first and she was early, as with many of my cousins.''

\end{quote}

Reports like these are called {\bf anecdotal evidence} because they
are based on data that is unpublished and usually personal.  In casual
conversation, there is nothing wrong with anecdotes, so I don't mean
to pick on the people I quoted.

But we might want evidence that is more persuasive and
an answer that is more reliable.  By those standards, anecdotal
evidence usually fails, because:

\begin{description}

\item[Small number of observations:] If the gestation period is longer
  for first babies, the difference is probably small compared to the
  natural variation.  In that case, we might have to compare a large
  number of pregnancies to be sure there is really a difference (or
  not).

\item[Selection bias:] People who join a discussion of this question
  might be interested because their first babies were late.  In that
  case the process of selecting data would bias the results.

\item[Confirmation bias:] People who believe the claim might be more
  likely to contribute examples that confirm it.  People who doubt the
  claim are more likely to cite counterexamples.

\item[Inaccuracy:] Anecotes are often personal stories that are
  (deliberately or not) misremembered, misrepresented, repeated
  inaccurately, etc.

\end{description}

So how can we do better?

\section{A statistical approach}

To address the limitations of anecdotes, we will use the tools
of statistics, which include:

\begin{description}

\item[Data collection:] We will use data from a large national survey
that was designed explicitly with the goal of generating statistically
valid inferences about the U.S. population.

\item[Exploratory data analysis:] We will start by exploring the dataset
to get a sense of what questions were asked, what form the answers
are in, and what limitations we might have to address.

\item[Descriptive statistics:] We will generate statistics that summarize
large datasets concisely.

\item[Hypothesis testing:] Where we see apparently effects (like a
difference between two groups), we will evaluate whether the effect
is likely to be real, or whether it might have happened by chance.

\item[Estimation:] We will use measurements in the dataset to estimate
characteristics of the general population.

\end{description}

By performing these steps with care to avoid common pitfalls, we can
reach conclusions that are more justifiable and more likely to be
correct.


\section{The National Survey of Family Growth}

Since 1973 the U.S. Centers for Disease Control and Prevention (CDC)
have conducted the National Survey of Family Growth (NSFG),
which is intended to gather ``information on family life, marriage and
divorce, pregnancy, infertility, use of contraception, and men's and
women's health. The survey results are used ... to plan health services and
health education programs, and to do statistical studies of families,
fertility, and health.''\footnote{See
  \url{cdc.gov/nchs/nsfg.htm}.}

We will use data collected by this survey to investigate whether first
babies tend to come late, and other questions.  In order to use this
data effectively, we have to understand the design of the study.

The NSFG is a {\bf cross-sectional} study, which means that it
captures a snapshot of a group at a point in time.  The most
common alternative is a {\bf longitudinal} study, which observes a
group repeatedly over a period of time.

The NSFG has been conducted seven times; each deployment is called
a {\bf cycle}.  We will be using data from Cycle 6, which was
conducted from January 2002 to March 2003.

The goal of the survey is to draw conclusions about a
{\bf population}; the target population of the NSFG is people in
the United States aged 15-44.

The people who participate in a survey are called {\bf respondents}.
In general, cross-sectional studies are meant to be {\bf
  representative}, which means that every member of the target
population has an equal chance of participating.  Of course that ideal
is hard to achieve in practice, but people who conduct surveys come as
close as they can.

The NSFG not representative; instead it is deliberately {\bf
  oversampled}.  The designers of the study recruited three
groups---Hispanics, African-Americans and teenagers---at rates higher
than their representation in the U.S. population.
The reason for oversampling is to make sure that the number of
respondents in each of these groups is large enough to draw valid
statistical inferences.

Of course, the drawback of oversampling is that it is not as easy
to draw conclusions about the general population based on statistics
from the survey.  We will come back to this point later.

\begin{ex}

Although the NSFG has been conducted seven times, it is not a
longitudinal study.  Read the Wikipedia pages
\url{wikipedia.org/wiki/Cross-sectional_study}
and
\url{wikipedia.org/wiki/Longitudinal_study}
to make sure you understand why not.

\end{ex}

\begin{ex}

In this exercise, you will download data from the NSFG and do some
exploratory data analysis.

\begin{enumerate}

\item Go to \url{www.cdc.gov/nchs/nsfg/nsfg_cycle6.htm} and find the
  section heading ``Downloadable Data Files.''  If you click on the
  file named ``Female Respondent Data File,'' you will be taken to the
  ``Data User's Agreement.''  Read the terms of this agreement and
  click ``I accept these terms'' (assuming that you do).

\item Download the files named {\tt 2002FemResp.dat} and {\tt
  2002FemPreg.dat}.  The first is the respondent file, which contains
  one record (line of text) for each of the 7,643 female respondents.
  The second file contains one record for each pregnancy reported by a
  respondent.

\item Online documentation of the survey is at
  \url{nsfg.icpsr.umich.edu/cocoon/WebDocs/NSFG/public/index.htm}.

\item The web page for this book provides code to process the data
  files from the NSFG.  Download \url{} and run it in the same
  directory you put the data files in.

\item 

\item

\end{enumerate}

\end{ex}


\begin{ex}

The best way to learn about statistics is to work on a project you are
interested in.  Is there a question like, ``Do first babies arrive
late,'' that you would like to investigate?

Think about questions you find personally interesting, or items of
conventional wisdom, or controversial topics, or questions that have
political consequences, and see if you can formulate a question that
lends itself to statistical inquiry.

Now start looking for datasets to help you address the
question.  Governments are good sources because data from
public research is often freely available.

Another way to find data is Wolfram Alpha, which is a curated
collection of good-quality datasets at \url{wolframalpha.com}.  But
results generated by Wolfram Alpha are subject to copyright
restrictions; you might want to check the terms before you commit
yourself.

Google and other search engines can also help you find data, but it
can be harder to evaluate the quality of resources on the web.

If it seems like someone has answered your question, you
should look closely to see whether the answer is justified.  You
might find flaws in the data or the analysis that make the
conclusion unreliable.  In that case you might perform a different
analysis of the same dataset, or look for a better source of data.

If you find a published paper that addresses your question, you
should be able to get the raw data.  Many authors make their data
available on the web, but for sensitive data you might have to
write to the authors, provide information about how you plan to use
the data, or agree to certain terms of use.  Be persistent!

\end{ex}

\printindex

\clearemptydoublepage
%\blankpage
%\blankpage
%\blankpage


\end{document}
