
\section{\module{plasTeX.Context} --- The \TeX\ Context\label{sec:context-api}}

\declaremodule{standard}{plasTeX.Context}
\modulesynopsis{The \TeX\ document context}

The \class{Context} class stores all of the information associated with
the currently running document.  This includes things like macros, counters, 
labels, references, etc.  The context also makes sure that localized macros
get popped off when processing leaves a macro or environment.  The context
of a document also has the power to create new counters, dimens, if commands,
macros, as well as change token category codes.

Each time a \class{TeX} object is instantiated, it will create its own 
context.  This context will load all of the base macros and initialize
all of the context information described above.


\subsection{Context Objects}

\begin{classdesc}{Context}{\optional{load}}
Instantiate a new context.

If the \var{load} argument is set to true, the context will load all of the
base macros defined in \plasTeX.  This includes all of the macros used in 
the standard \TeX\ and \LaTeX\ distributions.
\end{classdesc}

\begin{memberdesc}[Context]{contexts}
stack of all macro and category code collections currently in the document 
being processed.  The item at index 0 include the global macro set and 
default category codes.
\end{memberdesc}

\begin{memberdesc}[Context]{counters}
a dictionary of counters.
\end{memberdesc}

\begin{memberdesc}[Context]{currentlabel}
the object that is given the label when a \macro{label} macro is invoked.
\end{memberdesc}

\begin{memberdesc}[Context]{isMathMode}
boolean that specifies if we are currently in \TeX's math mode or not.
\end{memberdesc}

\begin{memberdesc}[Context]{labels}
a dictionary of labels and the objects that they refer to.
\end{memberdesc}



\begin{methoddesc}[Context]{addGlobal}{key, value}
add a macro \var{value} with name \var{key} to the global namespace. 
\end{methoddesc}

\begin{methoddesc}[Context]{addLocal}{key, value}
add a macro \var{value} with name \var{key} to the current namespace. 
\end{methoddesc}

\begin{methoddesc}[Context]{append}{\optional{context}}
same as \method{push()}
\end{methoddesc}

\begin{methoddesc}[Context]{catcode}{char, code}
set the category code for a character in the current scope.  \var{char}
is the character that will have its category code changed.  \var{code}
is the \TeX\ category code (0-15) to change it to.
\end{methoddesc}

\begin{methoddesc}[Context]{chardef}{name, num}
create a new \TeX\ chardef like \macro{chardef}.

\var{name} is the name of the command to create.

\var{num} is the character number to use.
\end{methoddesc}

\begin{methoddesc}[Context]{__getitem__}{key}
look through the stack of macros and return the one with the name \var{key}.
The return value is an \emph{instance} of the requested macro,
not a reference to the macro class.
This method allows you to use Python's dictionary syntax to retrieve
the item from the context as shown below.
\begin{verbatim}
tex.context['section']
\end{verbatim}
\end{methoddesc}

\begin{methoddesc}[Context]{importMacros}{context}
import macros from another context into the global namespace.  The argument,
\var{context}, must be a dictionary of macros.
\end{methoddesc}

\begin{methoddesc}[Context]{label}{label}
set the given label to the currently labelable object.  An object can 
only have one label associated with it.
\end{methoddesc}

\begin{methoddesc}[Context]{let}{dest, source}
create a new \TeX\ let like \macro{let}.

\var{dest} is the command sequence to create.

\var{source} is the token to set the command sequence equivalent to.

\textbf{Example}
\begin{verbatim}
c.let('bgroup', BeginGroup('{'))
\end{verbatim}
\end{methoddesc}

\begin{methoddesc}[Context]{loadBaseMacros}{}
imports all of the base macros defined by \plasTeX.  This includes all of
the macros specified by the \TeX\ and \LaTeX\ systems.
\end{methoddesc}

\begin{methoddesc}[Context]{loadLanguage}{language, document}
loads a language package to configure names such as \macro{figurename},
\macro{tablename}, etc.

\var{language} is a string containing the name of the language file to load.

\var{document} is the document object being processed.
\end{methoddesc}

\begin{methoddesc}[Context]{loadINIPackage}{inifile}
load an INI formatted package file (see section \ref{sec:packages} for 
more information).
\end{methoddesc}

\begin{methoddesc}[Context]{loadPackage}{tex, file, \optional{options}}
loads a \LaTeX\ package.  

\var{tex} is the \TeX\ processor to use in parsing the package content

\var{file} is the name of the package to load

\var{options} is a dictionary containing the options to pass to the package.
This generally comes from the optional argument on a \macro{usepackage}
or \macro{documentclass} macro.

The package being loaded by this method can be one of three type: 1)
a native \LaTeX\ package, 2) a Python package, or 3) an INI formatted file.
The Python version of the package is searched for first.  If it is found,
it is loaded and an INI version of the package is also loaded if it
exists.  If there is no Python version, the true \LaTeX\ version of the
package is loaded.  If there is an INI version of the package in the same
directory as the \LaTeX\ version, that file is loaded also.
\end{methoddesc}

\begin{methoddesc}[Context]{newcommand}{name\optional{, nargs\optional{, definition\optional{, opt}}}}
create a new \LaTeX\ command like \macro{newcommand}.

\var{name} is the name of the macro to create.

\var{nargs} is the number of arguments including optional arguments.

\var{definition} is a string containing the macro definition.

\var{opt} is a string containing the default optional value.

\textbf{Examples}
\begin{verbatim}
c.newcommand('bold', 1, r'\\textbf{#1}')
c.newcommand('foo', 2, r'{\\bf #1#2}', opt='myprefix')
\end{verbatim}
\end{methoddesc}

\begin{methoddesc}[Context]{newcount}{name\optional{, initial}}
create a new count like \macro{newcount}.
\end{methoddesc}

\begin{methoddesc}[Context]{newcounter}{name, \optional{resetby, initial, format}}
create a new counter like \macro{newcounter}.

\var{name} is the name of the counter to create.

\var{resetby} is the counter that, when incremented, will reset the 
new counter.

\var{initial} is the initial value for the counter.

\var{format} is the printed format of the counter.

In addition to creating a new counter macro, another macro corresponding
to the \macro{the\var{name}} is created which prints the value of the
counter just like in \LaTeX.
\end{methoddesc}

\begin{methoddesc}[Context]{newdef}{name\optional{, args\optional{, definition\optional{, local}}}}
create a new \TeX\ definition like \macro{def}.

\var{name} is the name of the definition to create.

\var{args} is a string containing the \TeX\ argument profile.

\var{definition} is a string containing the macro code to expand when the
definition is invoked.

\var{local} is a boolean that specifies that the definition should only
exist in the local scope.  The default value is true.

\textbf{Examples}
\begin{verbatim}
c.newdef('bold', '#1', '{\\bf #1}')
c.newdef('put', '(#1,#2)#3', '\\dostuff{#1}{#2}{#3}')
\end{verbatim}
\end{methoddesc}

\begin{methoddesc}[Context]{newdimen}{name\optional{, initial}}
create a new dimen like \macro{newdimen}.
\end{methoddesc}

\begin{methoddesc}[Context]{newenvironment}{name\optional{, nargs\optional{, definition\optional{, opt}}}}
create a new \LaTeX\ environment like \macro{newenvironment}.  This works
exactly like the \method{newcommand()} method, except that the 
\var{definition} argument is a two element tuple where the first element
is a string containing the macro content to expand at the \macro{begin},
and the second element is the macro content to expand at the \macro{end}.

\textbf{Example}
\begin{verbatim}
c.newenvironment('mylist', 0, (r'\\begin{itemize}', r'\\end{itemize}'))
\end{verbatim}
\end{methoddesc}

\begin{methoddesc}[Context]{newif}{name\optional{, initial}}
create a new if like \macro{newif}.  This also creates macros corresponding
to \macro{\var{name}true} and \macro{\var{name}false}.
\end{methoddesc}

\begin{methoddesc}[Context]{newmuskip}{name\optional{, initial}}
create a new muskip like \macro{newmuskip}.
\end{methoddesc}

\begin{methoddesc}[Context]{newskip}{name\optional{, initial}}
create a new skip like \macro{newskip}.
\end{methoddesc}

\begin{memberdesc}[Context]{packages}
a dictionary of \LaTeX\ packages.  The keys are the names of the packages.
The values are dictionaries containing the options that were specified
when the package was loaded.
\end{memberdesc}

\begin{methoddesc}[Context]{pop}{\optional{obj}}
pop the top scope off of the stack.  If \var{obj} is specified, continue
to pop scopes off of the context stack until the scope that was originally
added by \var{obj} is found.  
\end{methoddesc}

\begin{methoddesc}[Context]{push}{\optional{context}}
add a new scope to the stack.  If a macro instance \var{context} is specified, 
the new scope's namespace is given by that object.
\end{methoddesc}

\begin{methoddesc}[Context]{ref}{obj, label}
set up a reference for resolution.  

\var{obj} is the macro object that is doing the referencing.

\var{label} is the label of the node that \var{obj} is looking for.

If the item that \var{obj} is looking for has already been labeled, 
the \member{idref} attribute of \var{obj} is set to the abject.  
Otherwise, the reference is stored away to be resolved later. 
\end{methoddesc}

\begin{methoddesc}[Context]{setVerbatimCatcodes}{}
set the current set of category codes to the set used for the verbatim
environment.
\end{methoddesc}

\begin{methoddesc}[Context]{whichCode}{char}
return the character code that \var{char} belongs to.  The category
codes are the same codes used by \TeX\ and are defined in the 
\class{Token} class.  
\end{methoddesc}


