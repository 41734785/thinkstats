
\chapter{Understanding Macros and Packages\label{sec:macros}}

Macros and packages in \plasTeX\ live a dual life.  On one hand, macros
can be defined in \LaTeX\ files and expanded by \plasTeX\ itself.  On
the other hand, macros can also be implemented as Python classes.
Packages are the same way.  \plasTeX\ can handle some \LaTeX\ packages
natively.  Others may have to be implemented in Python.  In most
cases, both implementations work transparently together.  If you don't
define that many macros, and the ones that you do define are simple
or even of intermediate complexity, it's probably better to just let
\plasTeX\ handle them natively. However, 
there are some reasons that you may want to implement Python versions
of your macros:
\begin{itemize}
\item Python versions of macros are generally faster
\item You have more control over what gets inserted into the output document
\item You can store information in the document's \member{userdata}
    dictionary for use later 
\item You can prevent a macro from being expanded into primitive \LaTeX\
    commands, so that a custom renderer can be used on that node
\item Some macros just don't make sense in a \plasTeX\ document
\item Some macros are just too complicated for \plasTeX
\end{itemize}

If any of these reasons appeal to you, read the following sections on
how to implement macros and packages in \plasTeX.


\section{Defining Macros in \LaTeX}

Defining macros in \LaTeX\ using \plasTeX\ is no different than the way
you would normally define you macros; however, there is a trick that you
can use to improve you macros for \plasTeX, if needed.  While \plasTeX\
can handle fairly complicated macros, some macros might do things that
don't make sense in the context of a \plasTeX\ document, or they might
just be too complicated for the \plasTeX\ engine to handle.  In
cases such as these, you can use the \macro{ifplastex} construct.
As you may know in \TeX, you can define your own \macro{if} commands using
the \macro{newif} primitive.  There is an \macro{if} command called
\macro{ifplastex} built into the \plasTeX\ engine that is always set to
true.  In you document, you can define this command and set it to 
false (as far as \LaTeX\ is concerned) as follows.
\begin{verbatim}
\newif\ifplastex
\plastexfalse
\end{verbatim}

Now you can surround the portions of your macros that \plasTeX\ has 
trouble with, or even write alternative versions of the macro for 
\LaTeX\ and \plasTeX.  Here is an example.
\begin{verbatim}
\newcommand{\foo}[1]{
    \ifplastex\else\vspace*{0.25in}\fi
    \textbf{\Large{#1}}
    \ifplastex\else\vspace*{1in}\fi
}

\ifplastex
    \newenvironment{coolbox}{}{}
\else
    \newenvironment{coolbox}
        {fbox\bgroup\begin{minipage}{5in}}
        {\end{minipage}\egroup}
\fi
\end{verbatim}


\section{Defining Macros in Python}

Defining macros using Python classes (or, at least through Python interfaces)
is done in one of three ways: INI files, Python classes, and the document
context.  These three methods are described in the following sections.


\subsection{Python Classes\label{sec:macroclasses}}

Both \LaTeX\ command and environments can be implemented in Python
classes.  \plasTeX\ includes a base class for each one: \class{Command}
for commands and \class{Environment} for environments.  For the most
part, these two classes behave in the same way.  They both are 
responsible for parsing their arguments, organizing their child nodes,
incrementing counters, etc. much like their \LaTeX\ counterparts.
The Python macro class feature set is based on common \LaTeX\ 
conventions.  So if the \LaTeX\ macro you are implementing in Python
uses standard \LaTeX\ conventions, you job will be very easy.  If you
are doing unconventional operations, you will probably still succeed,
you just might have to do a little more work.

The three most important parts of the Python macro API are: 1) the 
\member{args} attribute, 2) the \method{invoke} method, and 3) the
\member{digest} method.  When writing your own macros, these are
used the most by far.

\subsubsection{The \member{args} Attribute}

The \member{args} attribute is a string attribute on the class 
that indicates what the arguments to the macro are.  In addition to
simply indicating the number of arguments, whether they are mandatory
or optional, and what characters surround the argument as in \LaTeX,
the \member{args} string also gives names to each of the argument
and can also indicate the content of the argument (i.e. int, float,
list, dictionary, string, etc.).  The names given to each argument
determine the key that the argument is stored under in the the
\member{attributes} dictionary of the class instance.  Below is a simple
example of a macro class.
\begin{verbatim}
from plasTeX import Command, Environment

class framebox(Command):
    """ \framebox[width][pos]{text} """
    args = '[ width ] [ pos ] text'
\end{verbatim}

In the \member{args} string of the \macro{framebox} macro, three arguments
are defined.  The first two are optional and the third one is mandatory.
Once each argument is parsed, in is put into the \member{attributes}
dictionary under the name given in the \member{args} string.  For example,
the \member{attributes} dictionary of an instance of \macro{framebox}
will have the keys ``width'', ``pos'', and ``text'' once it is parsed
and can be accessed in the usual Python way.
\begin{verbatim}
self.attributes['width']
self.attributes['pos']
self.attributes['text']
\end{verbatim}

In \plasTeX, any argument that isn't mandatory (i.e. no grouping characters
in the \member{args} string) is optional\footnote{While this isn't always true
when \LaTeX\ expands the macros, it will not cause any problems when
\plasTeX\ compiles the document because \plasTeX\ is less stringent.}.
This includes arguments surrounded by parentheses ((~)), square brackets
([~]), and angle brackets (<~>).  This also lets you combine multiple
versions of a command into one macro.  For example, the \macro{framebox}
command also has a form that looks like: 
\macro{framebox(x_dimen,y_dimen)[pos]\{text\}}.  This leads to the 
Python macro class in the following code sample that encompasses both 
forms.
\begin{verbatim}
from plasTeX import Command, Environment

class framebox(Command):
    """ 

    \framebox[width][pos]{text} or 
    \framebox(x_dimen,ydimen)[pos]{text} 

    """
    args = '( dimens ) [ width ] [ pos ] text'
\end{verbatim}
The only thing to keep in mind is that in the second form, the \var{pos}
attribute is going to end up under the \var{width} key in the 
\member{attributes} dictionary since it is the first argument in 
square brackets, but this can be fixed up in the \method{invoke} method
if needed.  Also, if an optional argument is not present on the 
macro, the value of that argument in the \member{attributes} dictionary
is set to \var{None}.

As mentioned earlier, it is also possible to convert arguments
to data types other than the default (a document fragment).  A list
of the available types is shown in the table below.
\begin{tableii}{l|p{4in}}{var}{Name}{Purpose}
\lineii{str}{expands all macros then sets the value of the argument
   in the \member{attributes} dictionary to the string content of the argument}
\lineii{chr}{same as `str'}
\lineii{char}{same as `str'}
\lineii{cs}{sets the attribute to an unexpanded control sequence}
\lineii{label}{expands all macros, converts the result to a string, then sets
    the current label to the object that is in the \member{currentlabel}
    attribute of the document context.  Generally, an object is put into the 
    \member{currentlabel} attribute if it incremented a counter when it
    was invoked.  The value stored in the \member{attributes} dictionary
    is the string value of the argument.}
\lineii{id}{same as `label'}
\lineii{idref}{expands all macros, converts the result to a string, 
    retrieves the object that was labeled by that value,
    then adds the labeled object to the \member{idref} dictionary under
    the name of the argument.
    This type of argument is used in commands like \macro{ref} that must
    reference other abjects.  The nice thing about `idref' is that it
    gives you a reference to the object itself which you can then use
    to retrieve any type of information from it such as the reference
    value, title, etc.  The value stored in the \member{attributes} 
    dictionary is the string value of the argument.}
\lineii{ref}{same as `idref'}
\lineii{nox}{just parses the argument, but doesn't expand the macros}
\lineii{list}{converts the argument to a Python list.  By default, the list item
    separator is a comma (,).  You can change the item separator in
    the args string by appending a set of parentheses surrounding the 
    separator character immediately after `list'.  For example, to
    specify a semi-colon separated list for an argument called ``foo''
    you would use the \member{args} string: ``foo:list(;)''.  It is also
    possible to cast the type of each item by appending another colon
    and the data type from this table that you want each item to be.
    However, you are limited to one data type for every item in the list.}
\lineii{dict}{converts the argument to a Python dictionary.  This is commonly
    used by arguments set up using \LaTeX's \file{keyval} package.
    By default, key/value pairs are separated by commas, although this
    character can be changed in the same way as the delimiter in the
    `list' type.  You can also cast each value of the dictionary using
    the same method as the `list' type.  In all cases, keys are converted
    to strings.}
\lineii{dimen}{reads a dimension and returns an instance of \class{dimen}}
\lineii{dimension}{same as `dimen'}
\lineii{length}{same as `dimen'}
\lineii{number}{reads an integer and returns a Python integer}
\lineii{count}{same as `number'}
\lineii{int}{same as `number'}
\lineii{float}{reads a decimal value and returns a Python float}
\lineii{double}{same as `float'}
\end{tableii}

There are also several argument types used for more low-level routines.
These don't parse the typical \LaTeX\ arguments, they are used for
the somewhat more free-form \TeX\ arguments.
\begin{tableii}{l|p{4in}}{var}{Name}{Purpose}
\lineii{Dimen}{reads a \TeX\ dimension and returns an instance of \class{dimen}}
\lineii{Length}{same as `Dimen'}
\lineii{Dimension}{same as `Dimen'}
\lineii{MuDimen}{reads a \TeX\ mu-dimension and returns an instance of \class{mudimen}}
\lineii{MuLength}{same as `MuDimen'}
\lineii{Glue}{reads a \TeX\ glue parameter and returns an instance of \class{glue}}
\lineii{Skip}{same as `MuLength'}
\lineii{Number}{reads a \TeX\ integer parameter and returns a Python integer}
\lineii{Int}{same as `Number'}
\lineii{Integer}{same as `Number'}
\lineii{Token}{reads an unexpanded token}
\lineii{Tok}{same as `Token'}
\lineii{XToken}{reads an expanded token}
\lineii{XTok}{same as `XToken'}
\lineii{Args}{reads tokens up to the first begin group (i.e. \{)}
\end{tableii}

To use one of the data types, simple append a colon (:) and the data
type name to the attribute name in the \member{args} string.
Going back to the \macro{framebox} example, the argument in 
parentheses would be better represented as a list of dimensions.
The \var{width} parameter is also a dimension, and the \var{pos} parameter
is a string.
\begin{verbatim}
from plasTeX import Command, Environment

class framebox(Command):
    """ 

    \framebox[width][pos]{text} or 
    \framebox(x_dimen,ydimen)[pos]{text} 

    """
    args = '( dimens:list:dimen ) [ width:dimen ] [ pos:chr ] text'
\end{verbatim}


\subsubsection{The \method{invoke} Method}

The \method{invoke} method is responsible for creating a new document
context, parsing the macro arguments, and incrementing counters.
In most cases, the default implementation will work just fine, but you
may want to do some extra processing of the macro arguments or 
counters before letting the parsing of the document proceed.  There
are actually several methods in the API that are called within the
scope of the \method{invoke} method: \method{preParse}, \method{preArgument},
\method{postArgument}, and \method{postParse}.

The order of execution is quite simple.  Before any arguments have been
parsed, the \method{preParse} method is called.  The \method{preArgument}
and \method{postArgument} methods are called before and after each
argument, respectively.  Then, after all arguments have been parsed, the
\method{postParse} method is called.  The default implementations of 
these methods handle the stepping of counters and setting the 
current labeled item in the document.  By default, macros that have
been ``starred'' (i.e. have a `*' before the arguments) do not 
increment the counter.  You can override this behavior in one of these
methods if you prefer.

The most common reason for overriding the \method{invoke} method is to
post-process the arguments in the \member{attributes} dictionary, or
add information to the instance.
For example, the \macro{color} command in \LaTeX's color package
could convert the \LaTeX\ color to the correct CSS format
and add it to the CSS style object.
\begin{verbatim}
from plasTeX import Command, Environment

def latex2htmlcolor(arg):
    if ',' in arg:
        red, green, blue = [float(x) for x in arg.split(',')]
        red = min(int(red * 255), 255)
        green = min(int(green * 255), 255)
        blue = min(int(blue * 255), 255)
    else:
        try:
            red = green = blue = float(arg)
        except ValueError:
            return arg.strip()
    return '#%.2X%.2X%.2X' % (red, green, blue)

class color(Environment):
    args = 'color:str'
    def invoke(self, tex):
        a = Environment.invoke(tex)
        self.style['color'] = latex2htmlcolor(a['color'])
\end{verbatim}

While simple things like attribute post-processing is the most common
use of the \method{invoke} method, you can do very advanced things
like changing category codes, and iterating over the tokens in the
\TeX\ processor directly like the \environment{verbatim} environment
does.

One other feature of the \method{invoke} method that may be of interest
is the return value.  Most \method{invoke} method implementations do
not return anything (or return \var{None}).  In this case, the macro
instance itself is sent to the output stream.  However, you can also
return a list of tokens.  If a list of tokens is returned, instead of
the macro instance, those tokens are inserted into the output stream.
This is useful if you don't want the macro instance to be part of the
output stream or document.  In this case, you can simply return an
empty list.


\subsubsection{The \method{digest} Method}

The \method{digest} method is responsible for converting the output
stream into the final document structure.  For commands, this generally
doesn't mean anything since they just consist of arguments which 
have already been parsed.  Environments, on the other hand, have a 
beginning and an ending which surround tokens that belong to that
environment.  In most cases, the tokens between the \macro{begin}
and \macro{end} need to be absorbed into the \member{childNodes}
list.

The default implementation of the \method{digest} method should work
for most macros, but there are instances where you may want to 
do some extra processing on the document structure.  For example,
the \macro{caption} command within \environment{figure}s and 
\environment{table}s uses the \method{digest} method to populate
the enclosing figure/table's \method{caption} attribute.
\begin{verbatim}
from plasTeX import Command, Environment

class Caption(Command):
    args = '[ toc ] self'

    def digest(self, tokens):
        res = Command.digest(self, tokens)

        # Look for the figure environment that we belong to 
        node = self.parentNode
        while node is not None and not isinstance(node, figure):
            node = node.parentNode

        # If the figure was found, populate the caption attribute
        if isinstance(node, figure):
            node.caption = self

        return res

class figure(Environment):
    args = '[ loc:str ]'
    caption = None
    class caption_(Caption):
        macroName = 'caption'
        counter = 'figure'
\end{verbatim}

More advanced uses of the \method{digest} method might be to construct
more complex document structures.  For example, tabular and array
structures in a document get converted from a simple list of tokens
to complex structures with lots of style information added (see section
\ref{sec:arrays}).  One simple example of a \method{digest} that 
does something extra is shown below.  It looks for the first 
node with the name ``item'' then bails out.
\begin{verbatim}
from plasTeX import Command, Environment

class toitem(Command):
    def digest(self, tokens):
        """ Throw away everything up to the first 'item' token """
        for tok in tokens:
            if tok.nodeName == 'item':
               # Put the item back into the stream
               tokens.push(tok)
               break
\end{verbatim}

One of the more advanced uses of the \method{digest} is on the sectioning
commands: \macro{section}, \macro{subsection}, etc.  The digest method
on sections absorb tokens based on the \member{level} attribute which
indicates the hierarchical level of the node.  When digested, each section
absorbs all tokens until it reaches a section that has a level that is
equal to or higher than its own level.  This creates the overall 
document structure as discussed in section \ref{sec:document}.


\subsubsection{Other Nifty Methods and Attributes}

There are many other attributes and methods on macros that can be used
to affect their behavior.  For a full listing, see the API documentation
in section \ref{sec:macros-api}.  Below are descriptions of some of the
more commonly used attributes and methods.

\paragraph{The \member{level} attribute}
The \member{level} attribute is an integer that indicates the 
hierarchical level of the node in the output document structure.  
The values of this attribute are taken from \LaTeX: \macro{part}
is -1, \macro{chapter} is 0, \macro{section} is 1, \macro{subsection}
is 2, etc.  To create your owne sectioning commands, you can either
subclass one of the existing sectioning macros, or simply set its
\member{level} attribute to the appropriate number.

\paragraph{The \member{macroName} attribute}
The \member{macroName} attribute is used when you are creating a
\LaTeX\ macro whose name is not a legal Python class name.  
For example, the macro \macro{@ifundefined} has a `@' in the name
which isn't legal in a Python class name.  In this case, you could define
the macro as shown below.
\begin{verbatim}
class ifundefined_(Command):
    macroName = '@ifundefined'
\end{verbatim}

\paragraph{The \member{counter} attribute}
The \member{counter} attribute associates a counter with the macro
class.  It is simply a string that contains the name of the counter.
Each time that an instance of the macro class is invoked, the
counter is incremented (unless the macro has a `*' argument).

\paragraph{The \member{ref} attribute}
The \member{ref} attribute contains the value normally returned
by the \macro{ref} command.

\paragraph{The \member{title} attribute}
The \member{title} attribute retrieves the ``title'' attribute from
the \member{attributes} dictionary.  This attribute is also overridable.

\paragraph{The \member{fullTitle} attribute}
The same as the \member{title} attribute, but also includes the 
counter value at the beginning.

\paragraph{The \member{tocEntry} attribute}
The \member{tocEntry} attribute retrieves the ``toc'' attribute from
the \member{attributes} dictionary.  This attribute is also overridable.

\paragraph{The \member{fullTocEntry} attribute}
The same as the \member{tocEntry} attribute, but also includes the 
counter value at the beginning.

\paragraph{The \member{style} attribute}
The \member{style} attribute is a CSS style object.  Essentially, this is
just a dictionary where the key is the CSS property name and the value 
is the CSS property value.  It has an attribute called \member{inline}
which contains an inline version of the CSS properties for use
in the style= attribute of HTML elements.

\paragraph{The \member{id} attribute}
This attribute contains a unique ID for the object.  If the object
was labeled by a \macro{label} command, the ID for the object will 
be that label; otherwise, an ID is generated.

\paragraph{The \member{source} attribute}
The \member{source} attribute contains the \LaTeX\ source representation
of the node and all of its contents.

\paragraph{The \method{currentSection} attribute}
The \member{currentSection} attribute contains the section that the
node belongs to.

\paragraph{The \method{expand} method}
The \method{expand} method is a thin wrapper around the \method{invoke}
method.  It simply invokes the macro and returns the result of 
expanding all of the tokens.  Unlike \method{invoke}, you will 
always get the expanded node (or nodes); you will not get a \var{None}
return value.

\paragraph{The \method{paragraphs} method}
The \method{paragraphs} method does the final processing of paragraphs
in a node's child nodes.  It makes sure that all content is wrapped
within paragraph nodes.  This method is generally called from the
\method{digest} method.


\subsection{INI Files\label{sec:inimacros}}

Using INI files is the simplest way of creating customized Python macro
classes.  It does require a little bit of knowledge of writing 
macros in Python classes (section \ref{sec:macroclasses}), but not 
much.  The only two pieces of information about Python macro classes
you need to know are 1) the \member{args} string format, and 2)
the superclass name (in most cases, you can simply use \class{Command}
or \class{Environment}).  The INI file features correspond to Python
macros in the following way.
\begin{tableii}{l|l}{}{INI File}{Python Macro Use}
\lineii{section name}{the Python class to inherit from}
\lineii{option name}{the name of the macro to create}
\lineii{option value}{the args string for the macro}
\end{tableii}

Here is an example of an INI file that defines several macros.
\begin{verbatim}
[Command]
; \program{ self }
program=self
; \programopt{ self }
programopt=self

[Environment]
; \begin{methoddesc}[ classname ]{ name  { args } ... \end{methoddesc}
methoddesc=[ classname ] name args
; \begin{memberdesc}[ classname ]{ name  { args } ... \end{memberdesc}
memberdesc=[ classname ] name args

[section]
; \headi( options:dict )[ toc ]{ title }
headi=( options:dict ) [ toc ] title

[subsection]
; \headii( options:dict )[ toc ]{ title }
headii=( options:dict ) [ toc ] title
\end{verbatim}

In the INI file above, six macro are being defined.  \macro{program} and
\macro{programopt} both inherit from \class{Command}, the generic 
\LaTeX\ macro superclass.  They also both take a single mandatory argument
called ``self.''  There are two environments defined also: 
\environment{methoddesc} and \environment{memberdesc}.  Each of these
has three arguments where the first argument is optional.  The 
last two macros actually inherit from standard \LaTeX\ sectioning commands.
They add an option, surrounded by parentheses, to the options that
\macro{section} and \macro{subsection} already had defined.

INI versions of \plasTeX\ packages are loaded much in the same way as
Python \plasTeX\ packages.  For details on how packages are loaded, see
section \ref{sec:packages}.


\subsection{The Document Context\label{sec:contextmacros}}

It is possible to define commands using the same interface that is used
by the \plasTeX\ engine itself.  This interface belongs to the 
\class{Context} object (usually accessed through the document object's
\member{context} attribute).  Defining commands using the context object
is generally done in the \function{ProcessOptions} function of a 
package.  The following methods of the context object create new commands.
\begin{tableii}{l|p{4in}}{method}{Method}{Purpose}
\lineii{newcounter}{creates a new counter, and also creates a command
    called \macro{the{\it counter}} which generates the formatted version
    of the counter.  This macro corresponds to the \macro{newcounter} 
    macro in \LaTeX.}
\lineii{newcount}{corresponds to \TeX's \macro{newcount} command.}
\lineii{newdimen}{corresponds to \TeX's \macro{newdimen} command.}
\lineii{newskip}{corresponds to \TeX's \macro{newskip} command.}
\lineii{newmuskip}{corresponds to \TeX's \macro{newmuskip} command.}
\lineii{newif}{corresponds to \TeX's \macro{newif} command.  This 
    command also generates macros for \macro{{\it ifcommand}true} and
    \macro{{\it ifcommand}false}.}
\lineii{newcommand}{corresponds to \LaTeX's \macro{newcommand} macro.}
\lineii{newenvironment}{corresponds to \LaTeX's \macro{newenvironment} macro.}
\lineii{newdef}{corresponds to \TeX's \macro{def} command.}
\lineii{chardef}{corresponds to \TeX's \macro{chardef} command.}
\end{tableii}

\note{Since many of these methods accept strings containing \LaTeX\ 
markup, you need to remember that the category codes of some characters
can be changed during processing.  If you are defining macros using
these methods in the \function{ProcessOptions} function in a package,
you should be safe since this function is executed in the preamble of the 
document where category codes are not changed frequently.  However, if
you define a macro with this interface in a context where the category
codes are not set to the default values, you will have to adjust the 
markup in your macros accordingly.}

Below is an example of using this interface within the context of a package
to define some commands.  For the full usage of these methods see the
API documentation of the \class{Context} object in section 
\ref{sec:context-api}.
\begin{verbatim}
def ProcessOptions(options, document):
    context = document.context

    # Create some counters
    context.newcounter('secnumdepth', initial=3)
    context.newcounter('tocdepth', initial=2)

    # \newcommand{\config}[2][general]{\textbf{#2:#1}
    context.newcommand('config', 2, r'\textbf{#2:#1}', opt='general')

    # \newenvironment{note}{\textbf{Note:}}{}
    context.newenvironment('note', 0, (r'\textbf{Note:}', r''))
\end{verbatim}


\section{Packages\label{sec:packages}}

Packages in \plasTeX\ are loaded in one of three ways: standard \LaTeX\
package, Python package, and INI file.  \LaTeX\ packages are loaded in
much the same way that \LaTeX\ itself loads packages.  The \program{kpsewhich}
program is used to locate the requested file which can be either in
the search path of your \LaTeX\ distribution or in one of the directories
specified in the \environment{TEXINPUTS} environment variable.
\plasTeX\ read the file and expand the macros therein just as \LaTeX\
would do.

Python packages are located using Python's search path.  This includes 
all directories listed in \var{sys.path} as well as those listed in
the \environment{PYTHONPATH} environment variable. After a package is
loaded, it is checked to see if there is a function called
\function{ProcessOptions} in its namespace.  If there is, that function
is called with two arguments: 1) the dictionary of options that were
specified when loading the package, and 2) the document object that is
currently being processed.  This function allows you to make adjustments
to the loaded macros based on the options specified, and define new
commands in the document's context (see section \ref{sec:contextmacros} 
for more information).  Of course, you can also define Python based
macros (section \ref{sec:macroclasses}) in the Python package as well.

The last type of packages is based on the INI file format.  This 
format is discussed in more detail in section \ref{sec:inimacros}.
INI formatted packages are loaded in conjunction with a \LaTeX\ or
Python package.  When a package is loaded, an INI file with the same
basename is searched for in the same director as the package.  If it
exists, it is loaded as well.  For example, if you had a package
called \file{python.sty} and a file called \file{python.ini} in the same
package directory, \file{python.sty} would be loaded first,
then \file{python.ini} would be loaded.  The same operation applies 
for Python based packages.
