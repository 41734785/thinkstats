
\section{\module{plasTeX.ConfigManager} --- \plasTeX\ Configuration}

\declaremodule{standard}{plasTeX.ConfigManager}
\modulesynopsis{\plasTeX's configuration management system}

The configuration system in \plasTeX\ that parses the command-line options
and configuration files is very flexible.  While many options are setup
by the \plasTeX\ framework, it is possible for you to add your own options.
This is useful if you have macros that may need to be configured by 
configurable options, or if you write a renderer that surfaces special options
to control it.

The config files that \class{ConfigManager} supports are standard INI-style
files.  This is the same format supported by Python's \class{ConfigParser}.
However, this API has been extended with some dictionary-like behaviors
to make it more Python friendly.

In addition to the config files, \class{ConfigManager} can also parse
command-line options and merge the options from the command-line into
the options set by the given config files.  In fact, when adding options
to a \class{ConfigManager}, you specify both how they appear in the config
file as well as how they appear on the command-line.  Below is a basic 
example.

\begin{verbatim}
from plasTeX.ConfigManager import *
c = ConfigManager()

# Create a new section in the config file.  This corresponds to the
# [ sectionname ] sections in an INI file.  The returned value is 
# a reference to the new section
d = c.add_section('debugging')

# Add an option to the 'debugging' section called 'verbose'.
# This corresponds to the config file setting:
#
# [debugging]
# verbose = no
#
d['verbose'] = BooleanOption(
    """ Increase level of debugging information """,
    options = '-v --verbose !-q !--quiet',
    default = False,
)

# Read system-level config file
c.read('/etc/myconfig.ini')

# Read user-level config file
c.read('~/myconfig.ini')

# Parse the current command-line arguments
opts, args = c.getopt(sys.argv[1:])

# Print the value of the 'verbose' option in the 'debugging' section
print c['debugging']['verbose']
\end{verbatim}

One interesting thing to note about retrieving values from a 
\class{ConfigManager} is that you get the value of the option
rather than the option instance that you put in.  For example, in the
code above.  A \class{BooleanOption} in put into the `verbose' option
slot, but when it is retrieved in the \function{print} statement at
the end, it prints out a boolean value.  This is true of all option
types.  You can access the option instance in the \member{data} attribute
of the section (e.g. \code{c['debugging'].data['verbose']}).


\subsection{ConfigManager Objects}

\begin{classdesc}{ConfigManager}{defaults=\{~\}}
Instantiate a configuration class for \plasTeX\ that parses the command-line options
as well as reads the config files.  

The optional argument, \var{defaults},
is a dictionary of default values for the configuration object.  These 
values are used if a value is not found in the requested section.
\end{classdesc}

\begin{methoddesc}[ConfigManager]{__add__}{other}
merge items from another \class{ConfigManager}.  This allows you to add
\class{ConfigManager} instances with syntax like: config + other. 
This operation will modify the original instance.
\end{methoddesc}

\begin{methoddesc}[ConfigManager]{add_section}{name}
create a new section in the configuration with the given name.  This 
name is the name used for the section heading in the INI file (i.e. the
name used within square brackets (\lbrack~\rbrack) to start a section).
The return value of this method is a reference to the newly created section.  
\end{methoddesc}

\begin{methoddesc}[ConfigManager]{categories}{}
return the dictionary of categories
\end{methoddesc}

\begin{methoddesc}[ConfigManager]{copy}{}
return a deep copy of the configuration
\end{methoddesc}

\begin{methoddesc}[ConfigManager]{defaults}{}
return the dictionary of default values
\end{methoddesc}

\begin{methoddesc}[ConfigManager]{read}{filenames}
read configuration data contained in files specified by \var{filenames}.
Files that cannot be opened are silently ignored.  This is designed so that
you can specify a list of potential configuration file locations (e.g.
current directory, user's home directory, system directory), and all 
existing configuration files in the list will be read.  A single filename
may also be given.
\end{methoddesc}

\begin{methoddesc}[ConfigManager]{get}{section, option, raw=0, vars=\{~\}}
retrieve the value of \var{option} from the section \var{section}.
Setting \var{raw} to true prevents any string interpolation from occurring
in that value.  \var{vars} is a dictionary of addition value to use 
when interpolating values into the option.

\note{You can alsouse the alternative dictionary syntax: config[section].get(option).}
\end{methoddesc}

\begin{methoddesc}[ConfigManager]{getboolean}{section, option}
retrieve the specified value and cast it to a boolean
\end{methoddesc}

\begin{methoddesc}[ConfigManager]{get_category}{key}
return the title of the given category
\end{methoddesc}

\begin{methoddesc}[ConfigManager]{getfloat}{section, option}
retrieve the specified value and cast it to a float
\end{methoddesc}

\begin{methoddesc}[ConfigManager]{getint}{section, option}
retrieve the specified value and cast it to and integer
\end{methoddesc}

\begin{methoddesc}[ConfigManager]{get_opt}{section, option}
return the option value with any leading and trailing quotes removed
\end{methoddesc}

\begin{methoddesc}[ConfigManager]{getopt}{args=None, merge=True}
parse the command-line options.  If \var{args} is not given, the 
args are parsed from \code{sys.argv[1:]}.  If \var{merge} is set to false,
then the options are not merged into the configuration.  The return value
is a two element tuple.  The first value is a list of parsed options
in the form \code{(option, value)}, and the second value is the list of
arguments.
\end{methoddesc}

\begin{methoddesc}[ConfigManager]{get_optlist}{section, option, delim=','}
return the option value as a list using \var{delim} as the delimiter
\end{methoddesc}

\begin{methoddesc}[ConfigManager]{getraw}{section, option}
return the raw (i.e. un-interpolated) value of the option
\end{methoddesc}

\begin{methoddesc}[ConfigManager]{has_category}{key, title}
add a category to group options when printing the command-line help.
Command-line options can be grouped into categories to make options easier
to find when printing the usage message for a program.  Categories consist
of two pieces: 1) the name, and 2) the title.  The name is the key in
the category dictionary and is the name used when specifying which category
an option belongs to.  The title is the actual text that you see as a 
section header when printing the usage message.
\end{methoddesc}

\begin{methoddesc}[ConfigManager]{has_option}{section, name}
return a boolean indicating whether or not an option with the given name exists
in the given section
\end{methoddesc}

\begin{methoddesc}[ConfigManager]{has_section}{name}
return a boolean indicating whether or not a section with the given name exists
\end{methoddesc}

\begin{methoddesc}[ConfigManager]{__iadd__}{other}
merge items from another \class{ConfigManager}.  This allows you to add
\class{ConfigManager} instances with syntax like: config += other.
\end{methoddesc}

\begin{methoddesc}[ConfigManager]{options}{name}
return a list of configured option names within a section.  Options are all
of the settings of a configuration file within a section (i.e. the lines that 
start with `optionname=').
\end{methoddesc}

\begin{methoddesc}[ConfigManager]{__radd__}{other}
merge items from another \class{ConfigManager}.  This allows you to add
\class{ConfigManager} instances with syntax like: other + config.
This operation will modify the original instance.
\end{methoddesc}

\begin{methoddesc}[ConfigManager]{readfp}{fp, filename=None}
like \method{read()}, but the argument is a file object.  The optional
\var{filename} argument is used for printing error messages.
\end{methoddesc}

\begin{methoddesc}[ConfigManager]{remove_option}{section, option}
remove the specified option from the given section
\end{methoddesc}

\begin{methoddesc}[ConfigManager]{remove_section}{section}
remove the specified section
\end{methoddesc}

\begin{methoddesc}[ConfigManager]{__repr__}{}
return the configuration as an INI formatted string; this also includes 
options that were set from Python code.
\end{methoddesc}

\begin{methoddesc}[ConfigManager]{sections}{}
return a list of all section names in the configuration
\end{methoddesc}

\begin{methoddesc}[ConfigManager]{set}{section, option, value}
set the value of an option
\end{methoddesc}

\begin{methoddesc}[ConfigManager]{__str__}{}
return the configuration as an INI formatted string; however, do not 
include options that were set from Python code.
\end{methoddesc}

\begin{methoddesc}[ConfigManager]{to_string}{source=...}
return the configuration as an INI formatted string.  The \var{source}
option indicates which source of information should be included in
the resulting INI file.  The possible values are:
\begin{tableii}{l|l}{var}{Name}{Description}
\lineii{COMMANDLINE}{set from a command-line option}
\lineii{CONFIGFILE}{set from a configuration file}
\lineii{BUILTIN}{set from Python code}
\lineii{ENVIRONMENT}{set from an environment variable}
\end{tableii}
\end{methoddesc}

\begin{methoddesc}[ConfigManager]{write}{fp}
write the configuration as an INI formatted string to the given file object
\end{methoddesc}

\begin{methoddesc}[ConfigManager]{usage}{categories=[]}
print the descriptions of all command-line options.  If \var{categories}
is specified, only the command-line options from those categories is printed.
\end{methoddesc}


\subsection{ConfigSection Objects}

\begin{classdesc}{ConfigSection}{name, data=\{~\}}
Instantiate a \class{ConfigSection} object.  

\var{name} is the name of the section.

\var{data}, if specified, is the dictionary of data to initalize
the section contents with.
\end{classdesc}

\class{ConfigSection} objects are rarely instantiated manually.  They 
are generally created using the \class{ConfigManager} API (either the 
direct methods or the Python dictionary syntax).

\begin{memberdesc}[ConfigSection]{data}
dictionary that contains the option instances.  This is only accessed
if you want to retrieve the real option instances.  Normally, you would
use standard dictionary key access syntax on the section itself to
retrieve the option values.
\end{memberdesc}

\begin{memberdesc}[ConfigSection]{name}
the name given to the section.
\end{memberdesc}


\begin{methoddesc}[ConfigSection]{copy}{}
make a deep copy of the section object.
\end{methoddesc}

\begin{methoddesc}[ConfigSection]{defaults}{}
return the dictionary of default options associated with the parent
\class{ConfigManager}.
\end{methoddesc}

\begin{methoddesc}[ConfigManager]{get}{option, raw=0, vars=\{~\}}
retrieve the value of \var{option}.
Setting \var{raw} to true prevents any string interpolation from occurring
in that value.  \var{vars} is a dictionary of addition value to use 
when interpolating values into the option.

\note{You can alsouse the alternative dictionary syntax: section.get(option).}
\end{methoddesc}

\begin{methoddesc}[ConfigManager]{getboolean}{section, option}
retrieve the specified value and cast it to a boolean
\end{methoddesc}

\begin{methoddesc}[ConfigSection]{__getitem__}{key}
retrieve the value of an option.  This method allows you to use Python's 
dictionary syntax on a section as shown below.
\begin{verbatim}
# Print the value of the 'optionname' option
print mysection['optionname'] 
\end{verbatim}
\end{methoddesc}

\begin{methoddesc}[ConfigManager]{getint}{section, option}
retrieve the specified value and cast it to and integer
\end{methoddesc}

\begin{methoddesc}[ConfigManager]{getfloat}{section, option}
retrieve the specified value and cast it to a float
\end{methoddesc}

\begin{methoddesc}[ConfigManager]{getraw}{section, option}
return the raw (i.e. un-interpolated) value of the option
\end{methoddesc}

\begin{memberdesc}[ConfigSection]{parent}
a reference to the parent \class{ConfigManager} object.
\end{memberdesc}

\begin{methoddesc}[ConfigSection]{__repr__}{}
return a string containing an INI file representation of the section.
\end{methoddesc}

\begin{methoddesc}[ConfigSection]{set}{option, value}
create a new option or set an existing option with the name 
\var{option} and the value of \var{value}.
If the given value is already an option instance, it is simply inserted
into the section.  If it is not an option instance, an appropriate type
of option is chosen for the given type.
\end{methoddesc}

\begin{methoddesc}[ConfigSection]{__setitem__}{key, value}
create a new option or set an existing option with the name \var{key} and
the value of \var{value}.  This method allows you to use Python's 
dictionary syntax to set options as shown below.
\begin{verbatim}
# Create a new option called 'optionname'
mysection['optionname'] = 10
\end{verbatim}
\end{methoddesc}

\begin{methoddesc}[ConfigSection]{__str__}{}
return a string containing an INI file representation of the section.
Options set from Python code are not included in this representation.
\end{methoddesc}

\begin{methoddesc}[ConfigSection]{to_string}{\optional{source}}
return a string containing an INI file representation of the section.
The \var{source} option allows you to only display options from 
certain sources.  See the \method{ConfigManager.source()} method for more
information.
\end{methoddesc}


\subsection{Configuration Option Types}

There are several option types that should cover just about any type 
of command-line and configuration option that you may have.  However,
in the spirit of object-orientedness, you can, of course, subclass
one of these and create your own types.  \class{GenericOption} is the 
base class for all options.  It contains all of the underlying framework
for options, but should never be instantiated directly.  Only subclasses
should be instantiated.

\begin{classdesc}{GenericOption}{
    \optional{docsTring, options, default, optional, values, category,
              callback, synopsis, environ, registry, mandatory,
              name, source}
}
Declare a command line option.

Instances of subclasses of \class{GenericOption} must be placed in
a \class{ConfigManager} instance to be used.  See the documentation for
\class{ConfigManager} for more details.

\var{docstring} is a string in the format of Python documentation
   strings that describes the option and its usage.  The first
   line is assumed to be a one-line summary for the option.
   The following paragraphs are assumed to be a complete description
   of the option.  You can give a paragraph with the label
   'Valid Values:' that contains a short description of the
   values that are valid for the current option.  If this
   paragraph exists and an error is encountered while validating
   the option, this paragraph will be printed instead of the
   somewhat generic error message for that option type.

\var{options} is a string containing all possible variants of the
   option.  All variants should contain the '-', '--', etc. at
   the beginning.  For boolean options, the option can be preceded
   by a '!' to mean that the option should be turned OFF rather
   than ON which is the default.

\var{default} is a value for the option to take if it isn't specified
   on the command line

\var{optional} is a value for the option if it is given without a value.
   This is only used for options that normally take a value,
   but you also want a default that indicates that the option
   was given without a value.

\var{values} defines valid values for the option.  This argument can take
   the following forms:
        

\begin{tableii}{l|p{4in}}{var}{Type}{Description}
   \lineii{single value}{for \class{StringOption} this this is a string, for
      \class{IntegerOption} this is an integer, for \class{FloatOption} this is
      a float.  The single value mode is most useful when the 
      value is a regular expression.  For example, to specify
      that a \class{StringOption} must be a string of characters followed
      by a digit, 'values' would be set to \code{re.compile(r'{\textbackslash}w+{\textbackslash}d')}.}
   \lineii{range of values}{a two element list can be given to specify
      the endpoints of a range of valid values.  This is probably
      most useful on \class{IntegerOption} and \class{FloatOption}.  
      For example, to specify that an IntegerOption can only take the values
      from 0 to 10, 'values' would be set to [0,10]. 
      \note{This mode must \emph{always} use a Python list since using
            a tuple means something else entirely.}}
   \lineii{tuple of values}{a tuple of values can be used to specify 
      a complete list of valid values.  For example, to specify
      that an \class{IntegerOption} can take the values 1, 2, or 3, 'values'
      would be set to \code{(1,2,3)}.  If a string value can only take 
      the values, 'hi', 'bye', and any string of characters beginning
      with the letter 'z', 'values' would be set to 
      \code{('hi','bye',re.compile(r'z.*?'))}.
      \note{This mode must *always* use a Python tuple since using
            a list means something else entirely.}}
\end{tableii}

\var{category} is a category key which specifies which category the
   option belongs to (see the \class{ConfigManager} documentation on 
   how to create categories).

\var{callback} is a function to call after the value of the option has
   been validated.  This function will be called with the validated
   option value as its only argument.

\var{environ} is an environment variable to use as default value instead
   of specified value.  If the environment variable exists, it
   will be used for the default value instead of the specified value.

\var{registry} is a registry key to use as default value instead of
   specified value.  If the registry key exists, it will be used
   for the default value instead of the specified value.  A
   specified environment variable takes precedence over this value.
   \note{This is not implemented yet.}

\var{name} is a key used to get the option from its corresponding section.
   You do not need to specify this.  It will be set automatically
   when you put the option into the \class{ConfigManager} instance.

\var{mandatory} is a flag used to determine if the option itself is
   required to be present.  The idea of a "mandatory option" is
   a little strange, but I have seen it done.

\var{source} is a flag used to determine whether the option was set
   directly in the \class{ConfigManager} instance through Python,
   by a configuration file/command line option, etc.  You do not need
   to specify this, it will be set automatically during parsing.
   This flag should have the value of \var{BUILTIN}, \var{COMMANDLINE},
   \var{CONFIGFILE}, \var{ENVIRONMENT}, \var{REGISTRY}, or \var{CODE}.
\end{classdesc}

\begin{methoddesc}[GenericOption]{acceptsArgument}{}
return a boolean indicating whether or not the option accepts an argument on 
the command-line.  For example, boolean options do not accept an argument.
\end{methoddesc}

\begin{methoddesc}[GenericOption]{cast}{arg}
cast the given value to the appropriate type.
\end{methoddesc}

\begin{methoddesc}[GenericOption]{checkValues}{value}
check \var{value} against all possible valid values for the option.
If the value is invalid, raise an \exception{InvalidOptionError} exception.
\end{methoddesc}

\begin{methoddesc}[GenericOption]{clearValue}{}
reset the value of the option as if it had never been set.
\end{methoddesc}

\begin{methoddesc}[GenericOption]{getValue}{\optional{default}}
return the current value of the option.  If \var{default} is specified
and a value cannot be gotten from any source, it is returned.
\end{methoddesc}


\begin{methoddesc}[GenericOption]{__repr__}{}
return a string containing a command-line representation of the option and
its value.
\end{methoddesc}

\begin{methoddesc}[GenericOption]{requiresArgument}{}
return a boolean indicating whether or not the option requires an argument
on the command-line.
\end{methoddesc}

As mentioned previously, \class{GenericOption} is an abstract class 
(i.e. it should not be instantiated directly).  Only subclasses of 
\class{GenericOption} should be instantiated.  Below are some examples
of use of some of these subclasses, followed by the descriptions
of the subclasses themselves.

\begin{verbatim}
   BooleanOption(
      ''' Display help message ''',
      options = '--help -h',
      callback = usage,  # usage() function must exist prior to this
   )
\end{verbatim}

\begin{verbatim}
   BooleanOption(
      ''' Set verbosity ''',
      options = '-v --verbose !-q !--quiet',
   )
\end{verbatim}

\begin{verbatim}
   StringOption(
      '''
      IP address option

      This option accepts an IP address to connect to.

      Valid Values:
      '#.#.#.#' where # is a number from 1 to 255

      ''',
      options = '--ip-address',
      values = re.compile(r'\d{1,3}(\.\d{1,3}){3}'),
      default = '127.0.0.0',
      synopsis = '#.#.#.#',
      category = 'network',  # Assumes 'network' category exists
   )
\end{verbatim}

\begin{verbatim}
   IntegerOption(
      '''
      Number of seconds to wait before timing out

      Valid Values:
      positive integer

      ''',
      options = '--timeout -t',
      default = 300,
      values = [0,1e9],
      category = 'network',
   )
\end{verbatim}

\begin{verbatim}
   IntegerOption(
      '''
      Number of tries to connect to the host before giving up

      Valid Values:
      accepts 1, 2, or 3 retries

      ''',
      options = '--tries',
      default = 1,
      values = (1,2,3),
      category = 'network',
   )
\end{verbatim}

\begin{verbatim}
   StringOption(
      '''
      Nonsense option for example purposes only

      Valid Values:
      accepts 'hi', 'bye', or any string beginning with the letter 'z'

      ''',
      options = '--nonsense -n',
      default = 'hi',
      values = ('hi', 'bye', re.compile(r'z.*?')),
   )
\end{verbatim}

\begin{classdesc}{BooleanOption}{\optional{\class{GenericOption} arguments}}
Boolean options are simply options that allow you to specify an `on' or 
`off' state.  The accepted values for a boolean option in a config file
are `on', `off', `true', `false', `yes', `no', 0, and 1.  Boolean options on
the command-line do not take an argument; simply specifying the option
sets the state to true.

One interesting feature of boolean options is in specifying the command-line
options.  Since you cannot specify a value on the command-line (the existence
of the option indicates the state), there must be a way to set the state to
false.  This is done using the `not' operator (!).  When specifying the 
\var{options} argument of the constructor, if you prefix an command-line 
option with an exclamation point, the existence of that option indicates
a false state rather than a true state.  Below is an example of an \var{options}
value that has a way to turn debugging information on (\longprogramopt{debug}) 
or off (\longprogramopt{no-debug}).
\begin{verbatim}
BooleanOption( options = '--debug !--no-debug' )
\end{verbatim}
\end{classdesc}

\begin{classdesc}{CompoundOption}{\optional{\class{GenericOption} arguments}}
Compound options are options that contain multiple elements on the 
command-line.  They are simply groups of command-line arguments surrounded
by a pair of grouping characters (e.g. (~), [~], \{~\}, <~>).  This grouping
can contain anything including other command-line arguments.  However, 
all content between the grouping characters is unparsed.  This can be useful
if you have a program that wraps another program and you want to be able
to forward the wrapped program's options on.  An example of a compound option
used on the command-line is shown below.
\begin{verbatim}
# Capture the --diff-opts options to send to another program
mycommand --other-opt --diff-opts ( -ib --minimal ) file1 file2
\end{verbatim}
\end{classdesc}

\begin{classdesc}{CountedOption}{\optional{\class{GenericOption} arguments}}
A \class{CountedOption} is a boolean option that keeps track of how many
times it has been specified.  This is useful for options that control 
the verbosity of logging messages in a program where the number of times
an option is specified, the more logging information is printed.
\end{classdesc}

\begin{classdesc}{InputDirectoryOption}{\optional{\class{GenericOption} arguments}}
An \class{InputDirectoryOption} is an option that accepts a directory name
for input.
This directory name is checked to make sure that it exists and that it is
readable.  If it is not, a \exception{InvalidOptionError} exception is 
raised.
\end{classdesc}

\begin{classdesc}{OutputDirectoryOption}{\optional{\class{GenericOption} arguments}}
An \class{OutputDirectoryOption} is an option that accepts a directory name
for output.  If the directory exists, it is checked to make sure that it is
readable.  If it does not exist, it is created.
\end{classdesc}

\begin{classdesc}{InputFileOption}{\optional{\class{GenericOption} arguments}}
An \class{InputFileOption} is an option that accepts a file name for input.
The filename is checked to make sure that it exists and is readable.  If
it isn't, an \exception{InvalidOptionError} exception is raised.
\end{classdesc}

\begin{classdesc}{OutputFileOption}{\optional{\class{GenericOption} arguments}}
An \class{OutputFileOption} is an option that accepts a file name for output.
If the file exists, it is checked to make sure that it is writable.  
If a name contains a directory, the path is checked to make sure that it is 
writable.  If the directory does not exist, it is created. 
\end{classdesc}

\begin{classdesc}{FloatOption}{\optional{\class{GenericOption} arguments}}
A \class{FloatOption} is an option that accepts a floating point number.
\end{classdesc}

\begin{classdesc}{IntegerOption}{\optional{\class{GenericOption} arguments}}
An \class{IntegerOption} is an option that accepts an integer value.
\end{classdesc}

\begin{classdesc}{MultiOption}{\optional{\class{GenericOption} arguments, \optional{delim, range, template}}}
A \class{MultiOption} is an option that is intended to be used multiple times
on the command-line, or take a list of values.  Other options when specified
more than once simply overwrite the previous value.  \class{MultiOption}s
will append the new values to a list.

The delimiter used to separate multiple values is the comma (,).  A different
character can be specified in the \var{delim} argument.  

In addition, it is possible to specify the number of values that are legal
in the \var{range} argument.  The range argument takes a two element list.
The first element is the minimum number of times the argument is required.
The second element is the maximum number of times it is required.  You can
use a `*' (in quotes) to mean an infinite number.

You can cast each element in the list of values to a particular type by
using the \var{template} argument.  The \var{template} argument takes a
reference to the option class that you want the values to be converted to.
\end{classdesc}

\begin{classdesc}{StringOption}{\optional{\class{GenericOption} arguments}}
A \class{StringOption} is an option that accepts an arbitrary string.
\end{classdesc}
